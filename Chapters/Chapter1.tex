\chapter{Project Proposal}

In this first chapter we describe the main goal of this project, including the problem definition, justification, objectives, methodology and expected results.
\section{Problem definition}

Automatic Speech Recognition (ASR) has been one of the most investigated topic in Speech Processing \cite{Rabiner1993FundamentalsRecognition}. Many difficulties and open questions still remain unsolved, problems related to increasing dictionary size (more than 50.000 words), robust speech recognition (capability of recognition on noisy environments), speaker independence (capability for understand any speaker), spontaneous speech phenomena (false starts, out of vocabulary words, mumbling), distant speech recognition (the speaker is located far from the input source) and so on \cite{Trentin2001ARecognition}.

Although all this work, precision on ASR task related in Spanish Language does not exceed 70 percent of accuracy. Open Source resource are not enough to train robust models used in literature.

With this work we aim to minimize the gap between long studies languages like English and Spanish, providing effective implementations of frameworks, algorithms, models, techniques and methodologies to Spanish merging traditional frameworks with DL optimization.

The results are restricted to existing open corpus in Spanish Language and two proposal of new corpus: Open Speech Corpus \cite{QUOTE REQUIRED} and Libri Speech Spanish \cite{QOUTE REQUIRED}

\section{Justification}

ASR is a major field of research. Leader technology companies are leading investigation on modern systems building Intelligent Personal Assistants, as the case of Google with Google Home\cite{Li2017}, Microsoft with Cortana\cite{Xiong2017}, Apple with Siri and Amazon with Echo.

Applications of ASR technologies are broad and useful for numerous fields, including marketing, technical support, finances, personal assistants, monitoring, and more.

Speech represents a natural interface of communication for humans. Some psychologist consider that language is the key of human knowledge \cite{QOUTE REQUIRED} and speech still maintain as a key way to pass and store information.

\section{Objectives}

We present general and specific objectives in this section

\subsection{General objective}

Adapt Deep Learning relevant frameworks, algorithms, models, techniques and methodologies to improve Automatic Speech Recognition on Spanish Language.

\subsection{Specific objectives}

\begin{itemize}
    \item Select relevant algorithms, models, techniques and methodologies to improve ASR
    \item Collect Spanish resources to develop improvements on ASR
    \item Implement solutions based on DL to improve ASR precision
    \item Measure improvements of implementations achieved
\end{itemize}

\section{Methodology}

To accomplish each specific objectives the methodology  to be used start with a literature exploration of academic publications using hybrid techniques on ASR combining traditional frameworks and DL implementations. Then select most relevant resources to implement.

Simultaneously select open data to work in Spanish, including existing annotated corpus and adapting non-annotated corpus to work with tools selected.

With tools selected, implement required functionality and adapt to work in Spanish.

Finally, measure traditional techniques performance, response time, training time and resources required comparing to implementation proposed.

\section{Expected results}

With this thesis we expect two visible results:

\begin{itemize}
    \item An open framework to build ASR systems using DL and
    \item a larger corpus to work with ASR in Spanish
\end{itemize}




